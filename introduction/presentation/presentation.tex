\documentclass[slidestop]{beamer}

\author{Mihai Lefter}
\title{Python Programming}
\providecommand{\mySubTitle}{Introduction}
\providecommand{\myConference}{Work discussion}
\providecommand{\myDate}{27-11-2018}
\providecommand{\myGroup}{}
\providecommand{\myDepartment}{}
\providecommand{\myCenter}{}

\usetheme{lumc}

\usepackage{minted}
\usepackage{tikz}
\usepackage[many]{tcolorbox}

\definecolor{monokaibg}{HTML}{272822}

\newenvironment{ipython}
 {\begin{tcolorbox}[title=IPython,
                   title filled=false,
                   fonttitle=\scriptsize,
                   fontupper=\footnotesize,
                   enhanced,
                   colback=monokaibg,
                   drop small lifted shadow,
                   boxrule=0.1mm,
                   left=0.1cm,
                   arc=0mm,
                   colframe=black]}
 {\end{tcolorbox}
}

\definecolor{white}{rgb}{1,1,1}
\definecolor{mygreen}{rgb}{0,0.4,0}
\definecolor{light_gray}{rgb}{0.97,0.97,0.97}
\definecolor{mykey}{rgb}{0.117,0.403,0.713}

\tcbuselibrary{listings}
\newlength\inwd
\setlength\inwd{1.3cm}

\newcounter{ipythcntr}
\renewcommand{\theipythcntr}{\texttt{[\arabic{ipythcntr}]}}

\newtcblisting{pyin}[1][]{%
  sharp corners,
  enlarge left by=\inwd,
  width=\linewidth-\inwd,
  enhanced,
  boxrule=0pt,
  colback=light_gray,
  listing only,
  top=0pt,
  bottom=0pt,
  overlay={
    \node[
      anchor=north east,
      text width=\inwd,
      font=\footnotesize\ttfamily\color{mykey},
      inner ysep=2mm,
      inner xsep=0pt,
      outer sep=0pt
      ]
      at (frame.north west)
      {\refstepcounter{ipythcntr}\label{#1}In \theipythcntr:};
  }
  listing engine=listing,
  listing options={
    aboveskip=1pt,
    belowskip=1pt,
    basicstyle=\footnotesize\ttfamily,
    language=Python,
    keywordstyle=\color{mykey},
    showstringspaces=false,
    stringstyle=\color{mygreen},
    numbers=none,
    frame=none
  },
}
\newtcblisting{pyprint}{
  sharp corners,
  enlarge left by=\inwd,
  width=\linewidth-\inwd,
  enhanced,
  boxrule=0pt,
  colback=white,
  listing only,
  top=0pt,
  bottom=0pt,
  overlay={
    \node[
      anchor=north east,
      text width=\inwd,
      font=\footnotesize\ttfamily\color{mykey},
      inner ysep=2mm,
      inner xsep=0pt,
      outer sep=0pt
      ]
      at (frame.north west)
      {};
  }
  listing engine=listing,
  listing options={
      aboveskip=1pt,
      belowskip=1pt,
      basicstyle=\footnotesize\ttfamily,
      language=Python,
      keywordstyle=\color{mykey},
      showstringspaces=false,
      stringstyle=\color{mygreen},
      numbers=none,
      frame=none
    },
}
\newtcblisting{pyout}[1][\theipythcntr]{
  sharp corners,
  enlarge left by=\inwd,
  width=\linewidth-\inwd,
  enhanced,
  boxrule=0pt,
  colback=white,
  listing only,
  top=0pt,
  bottom=0pt,
  overlay={
    \node[
      anchor=north east,
      text width=\inwd,
      font=\footnotesize\ttfamily\color{mykey},
      inner ysep=2mm,
      inner xsep=0pt,
      outer sep=0pt
      ]
      at (frame.north west)
      {\setcounter{ipythcntr}{\value{ipythcntr}}Out#1:};
  }
  listing engine=listing,
  listing options={
      aboveskip=1pt,
      belowskip=1pt,
      basicstyle=\footnotesize\ttfamily,
      language=Python,
      keywordstyle=\color{mykey},
      showstringspaces=false,
      stringstyle=\color{mygreen},
      numbers=none,
      frame=none
    },
}



\begin{document}

% This disables the \pause command, handy in the editing phase.
%\renewcommand{\pause}{}

% Make the title slide.
\makeTitleSlide{\includegraphics[height=3.5cm]{../../images/Python.pdf}}

% First page of the presentation.
\section{Introduction}
\makeTableOfContents

\subsection{About the course}
\begin{pframe}
 \begin{itemize}
  \item Aimed at PhD students, Postdocs, researchers, analysts, ...
  \item Focus on:
  \begin{itemize}
   \item Basic understanding of Python.
   \item Programming as a tool to do your research.
   \item Slightly biased on bioinformatics.
  \end{itemize}
 \end{itemize}
\end{pframe}

\subsection{Hands on!}
\begin{pframe}
 Programming is fun!
 \begin{itemize}
  \item You only learn programming by doing it.
  \item Lecture format:
  \begin{itemize}
   \item Blended teaching + exercising.
  \end{itemize}
  \item Have your laptop open during the lessons.
  \item Repeat the code from the slides, play around with it.
  \item Do the session exercises.
  \item There will be a few assignments to submit.
 \end{itemize}
\end{pframe}

\subsection{Teachers}
\begin{pframe}
 \begin{itemize}
  \item Sander Bollen \\
    \url{a.h.b.bollen@lumc.nl}
  \item Jonathan Vis \\
    \url{j.k.vis@lumc.nl}
  \item Mark Santcroos \\
    \url{m.a.santcroos@lumc.nl}
  \item Guy Allard \\
    \url{w.g.Allard@lumc.nl}
  \item Mihai Lefter\\
    \url{m.lefter@lumc.nl}
 \end{itemize}
\end{pframe}

\subsection{Software requirements}
\begin{pframe}
 \begin{itemize}
  \item Anaconda:
  \begin{itemize}
   \item Python 3.7
   \item Comes with all that's required:
   \begin{itemize}
    \item Python interpreter.
    \item Jupyter Notebook.
    \item Libraries: NumPy, Panda, matplotlib, Bokeh, Biopython, ...
   \end{itemize}
   \item \href{http://docs.anaconda.com/anaconda/install/}{Installation instructions}.
  \end{itemize}
  \item Git.
 \end{itemize}
\end{pframe}

\subsection{Assignments}
\begin{pframe}
 \begin{itemize}
  \item We make use of GitHub Classroom.
  \begin{itemize}
   \item GitHub account required.
   \item Receive link with assignment repository.
  \end{itemize}
  \item Own forked repository to work on:
  \begin{itemize}
   \item Clone it.
   \item Code.
   \item Push it.
  \end{itemize}
  \item Direct file upload to repository is also possible.
 \end{itemize}
\end{pframe}

\subsection{Getting help}
\begin{pframe}
 \begin{itemize}
  \item Ask a teacher.
  \item If it's private, mail one of the teachers.
 \end{itemize}
\end{pframe}

\section{About Python}

\subsection{History}
\begin{pframe}
 \begin{itemize}
  \item Created early 90's by Guido van Rossem at CWI.
  \begin{itemize}
   \item Name: Monty Python.
  \end{itemize}
  \item General purpose, high-level programming language.
  \item Design is driven by code readability.
 \end{itemize}
\end{pframe}

\subsection{Features}
\begin{pframe}
 \begin{itemize}
  \item Interpreted, no separate compilation step needed.
  \item Imperative and object-oriented programming.
  \begin{itemize}
   \item And some functional programming.
  \end{itemize}
  \item Dynamic type system.
  \item Automatic memory management.
 \end{itemize}
 We'll come back to most of this.
\end{pframe}

\subsection{Why Python?}
\begin{pframe}
 \begin{itemize}
  \item Readable and low barrier to entry.
  \item Rich scientific libraries.
  \item Many other libraries available.
  \item Widely used with a large community.
 \end{itemize}
\end{pframe}

\subsection{Python 2 versus Python 3}
\begin{pframe}
 \begin{itemize}
  \item Python 3 is backwards incompatible.
  \item Some libraries don't support it yet.
  \item Python 2.7 is the last Python 2.
  \item Some Python 3 features are backported in Python 2.7.
  \item Python 2.7 will no longer .
 \end{itemize}
  We use Python 3.7 for the time being.
\end{pframe}

% \subsection{}
% \begin{pframe}
%  \begin{itemize}
%   \item
%  \end{itemize}
% \end{pframe}

\section{Python as a calculator}

\subsection{Integers}
\begin{pframe}
 \begin{pyin}
17
 \end{pyin}
 \begin{pyout}
17
 \end{pyout}
 \begin{pyin}
(17 + 4) * 2
 \end{pyin}
 \begin{pyout}
42
 \end{pyout}
\end{pframe}

\subsection{Floating point numbers}
\begin{pframe}
 \begin{pyin}
3.2 * 18 - 2.1
 \end{pyin}
 \begin{pyout}
55.5
 \end{pyout}
 \begin{pyin}
36. / 5
 \end{pyin}
 \begin{pyout}
7.2
 \end{pyout}

Scientific notation:
 \begin{pyin}
1.3e20 + 2
 \end{pyin}
 \begin{pyout}
1.3e+20
 \end{pyout}
 \begin{pyin}
1.3 * 10**20
 \end{pyin}
 \begin{pyout}
1.3e+20
 \end{pyout}

\end{pframe}

% \begin{pframe}
%   \begin{ipython}
%   \begin{minted}[breaklines=true,style=monokai]{python}
% a = 100
%   \end{minted}
%  \end{ipython}
% \end{pframe}


% Make the acknowledgements slide.
\makeAcknowledgementsSlide{
  \begin{tabular}{ll}
    \bf Department one  & \bf Other department\\
    Colleague one       & Someone\\
    Colleague two       & Someone else\\
    Colleague three     & Someone's supervisor\\
    Supervisor\\
  \end{tabular}
  \bigskip

  \begin{tabular}{llll}
    \includegraphics[height=1cm]{logos/lumc_logo_small} &
    \includegraphics[height=1cm]{logos/ul_logo_white} &
    \includegraphics[height=1cm]{logos/lumc_logo_small} &
    \includegraphics[height=1cm]{logos/ul_logo_white}
  \end{tabular}
}

\end{document}
