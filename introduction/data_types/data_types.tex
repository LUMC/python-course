\documentclass[aspectratio=1610,slidestop]{beamer}

\author{Mihai Lefter}
\title{Python Programming}
\providecommand{\mySubTitle}{Data Types}
\providecommand{\myConference}{Programming Course}
\providecommand{\myDate}{27-11-2018}
\providecommand{\myGroup}{}
\providecommand{\myDepartment}{}
\providecommand{\myCenter}{}

\usetheme{lumc}

\usepackage{minted}
\usepackage{tikz}
\usepackage[many]{tcolorbox}

\definecolor{monokaibg}{HTML}{272822}
\definecolor{emailc}{HTML}{1e90FF}
\definecolor{ipyout}{HTML}{F0FFF0}


\newenvironment{ipython}
 {\begin{tcolorbox}[title=IPython,
                   title filled=false,
                   fonttitle=\scriptsize,
                   fontupper=\footnotesize,
                   enhanced,
                   colback=monokaibg,
                   drop small lifted shadow,
                   boxrule=0.1mm,
                   left=0.1cm,
                   arc=0mm,
                   colframe=black]}
 {\end{tcolorbox}}


\newenvironment{terminal}
 {\begin{tcolorbox}[title=terminal,
                   title filled=false,
                   fonttitle=\scriptsize,
                   fontupper=\footnotesize,
                   enhanced,
                   colback=monokaibg,
                   drop small lifted shadow,
                   boxrule=0.1mm,
                   left=0.1cm,
                   arc=0mm,
                   colframe=black]}
 {\end{tcolorbox}}


\newcommand{\hrefcc}[2]{\textcolor{#1}{\href{#2}{#2}}}
\newcommand{\hrefc}[3]{\textcolor{#1}{\href{#2}{#3}}}

\newcounter{cntr}
\renewcommand{\thecntr}{\texttt{[\arabic{cntr}]}}

\newenvironment{pythonin}[1]
{\VerbatimEnvironment
  \begin{minipage}[t]{0.11\linewidth}
   \textcolor{green}{\texttt{{\refstepcounter{cntr}\label{#1}In \thecntr:}}}
  \end{minipage}%
  \begin{minipage}[t]{0.89\linewidth}%
  \begin{minted}[
    breaklines=true,style=monokai]{#1}}
 {\end{minted}
 \end{minipage}}

\newenvironment{pythonout}
{%
  \addtocounter{cntr}{-1}
  \begin{minipage}[t]{0.11\linewidth}
   \textcolor{red}{\texttt{{\refstepcounter{cntr}\label{#1}Out\thecntr:}}}
  \end{minipage}%
  \color{ipyout}%
  \ttfamily%
  \begin{minipage}[t]{0.89\linewidth}%
}
{\end{minipage}}

\newenvironment{pythonerr}[1]
{\VerbatimEnvironment
  \begin{minted}[
    breaklines=true,style=monokai]{#1}}
 {\end{minted}}

\begin{document}

% This disables the \pause command, handy in the editing phase.
%\renewcommand{\pause}{}

% Make the title slide.
\makeTitleSlide{\includegraphics[height=3.5cm]{../../images/Python.pdf}}

% First page of the presentation.
\section{Introduction}
\makeTableOfContents


\section{Sequence types}

\subsection{Lists}
\begin{pframe}
Mutable sequences of values.
 \begin{ipython}
  \begin{pythonin}{python}
l = [2, 5, 2, 3, 7]
  \end{pythonin}
  \\
  \begin{pythonin}{python}
type(l)
  \end{pythonin}
  \begin{pythonout}
list
  \end{pythonout}
 \end{ipython}
 \medskip

 Lists can be heterogeneous, but we typically don't use that.
 \begin{ipython}
  \begin{pythonin}{python}
a = 'spezi'
  \end{pythonin}
  \\
  \begin{pythonin}{python}
[3, 'abc', 1.3e20, [a, a, 2]]
  \end{pythonin}
  \begin{pythonout}
[3, 'abc', 1.3e+20, ['spezi', 'spezi', 2]]
  \end{pythonout}
 \end{ipython}
\end{pframe}


\subsection{Tuples}
\begin{pframe}
Immutable sequences of values.
 \begin{ipython}
  \begin{pythonin}{python}
t = 'white', 77, 1.5
  \end{pythonin}
  \\
  \begin{pythonin}{python}
type(t)
  \end{pythonin}
  \begin{pythonout}
tuple
  \end{pythonout}
 \\

 \begin{pythonin}{python}
color, width, scale = t
  \end{pythonin}
  \\
  \begin{pythonin}{python}
width
  \end{pythonin}
  \begin{pythonout}
77
  \end{pythonout}
 \end{ipython}
\end{pframe}


\subsection{Strings}
\begin{pframe}
Immutable sequences of characters.
 \begin{ipython}
  \begin{pythonin}{python}
'a string can be written in single quotes'
  \end{pythonin}
  \\
  \begin{pythonout}
'a string can be written in single quotes'
  \end{pythonout}
 \end{ipython}

Strings can also be written with double quotes, or over multiple lines with
triple-quotes.
 \begin{ipython}
  \begin{pythonin}{python}
"this makes it easier to use the ' character"
  \end{pythonin}
  \\
  \begin{pythonout}
"this makes it easier to use the ' character"
  \end{pythonout}
  \\

  \begin{pythonin}{python}
"""A multiline string.
You see? I continued after a blank line."""
  \end{pythonin}
  \begin{pythonout}
'A multiline string.\n\nYou see? I continued after a blank line.'
  \end{pythonout}
 \end{ipython}
\end{pframe}

\begin{pframe}
 A common operation is formatting strings using argument substitutions.
 \begin{ipython}
  \begin{pythonin}{python}
'{} times {} equals {:.2f}'.format('pi', 2, 6.283185307179586)
  \end{pythonin}
  \\
  \begin{pythonout}
'pi times 2 equals 6.28'
  \end{pythonout}
 \end{ipython}

 Accessing arguments by position or name is more readable.
 \begin{ipython}
  \begin{pythonin}{python}
'{1} times {0} equals {2:.2f}'.format('pi', 2, 6.283185307179586)
  \end{pythonin}
  \\
  \begin{pythonout}
'2 times pi equals 6.28'
  \end{pythonout}
  \\

  \begin{pythonin}{python}
'{number} times {amount} equals {result:.2f}'.format(number='pi', amount=2, result=6.283185307179586)
  \end{pythonin}
  \\
  \begin{pythonout}
'pi times 2 equals 6.28'
  \end{pythonout}
 \end{ipython}
\end{pframe}


\section{Common sequence operations}

\begin{pframe}
 All sequence types support: concatenation, membership/substring tests,
 indexing, and slicing.
 \\
 \begin{ipython}
  \begin{pythonin}{python}
[1, 2, 3] + [4, 5, 6]
  \end{pythonin}
  \\
  \begin{pythonout}
[1, 2, 3, 4, 5, 6]
  \end{pythonout}
  \\

  \begin{pythonin}{python}
'bier' in 'we drinken bier vanaf half 5'
  \end{pythonin}
  \\
  \begin{pythonout}
True
  \end{pythonout}
  \\

  \begin{pythonin}{python}
'abcdefghijkl'[5]
  \end{pythonin}
  \\
  \begin{pythonout}
'f'
  \end{pythonout}
 \end{ipython}
\end{pframe}


\subsection{Slicing}
\begin{pframe}
 Slice \mintinline{python}{s} from \mintinline{python}{i} to
 \mintinline{python}{j} with \mintinline{python}{s[i:j]}.
 \begin{ipython}
  \begin{pythonin}{python}
'abcdefghijkl'[4:8]
  \end{pythonin}
  \\
  \begin{pythonout}
'efgh'
  \end{pythonout}
  \\

  \begin{pythonin}{python}
'abcdefghijkl'[:3]
  \end{pythonin}
  \\
  \begin{pythonout}
'abc'
  \end{pythonout}
 \end{ipython}

 We can also define the step \mintinline{python}{k} with
 \mintinline{python}{s[i:j:k]}.
 \begin{ipython}
  \begin{pythonin}{python}
'abcdefghijkl'[7:3:-1]
  \end{pythonin}
  \\
  \begin{pythonout}
'hgfe'
  \end{pythonout}
 \end{ipython}
\end{pframe}


\subsection{Several helpful builtins}
\begin{pframe}
 \begin{ipython}
  \begin{pythonin}{python}
len('attacgataggcatccgt')
  \end{pythonin}
  \\
  \begin{pythonout}
18
  \end{pythonout}
  \\

  \begin{pythonin}{python}
max([17, 86, 34, 51])
  \end{pythonin}
  \\
  \begin{pythonout}
86
  \end{pythonout}
  \\

  \begin{pythonin}{python}
sum([17, 86, 34, 51])
  \end{pythonin}
  \\
  \begin{pythonout}
188
  \end{pythonout}
  \\

  \begin{pythonin}{python}
('atg', 22, True, 'atg').count('atg')
  \end{pythonin}
  \\
  \begin{pythonout}
2
  \end{pythonout}
 \end{ipython}
\end{pframe}


\subsection{More with lists}
\begin{pframe}
 We can replace, add, remove, reverse and sort items in-place.
 \begin{ipython}
  \begin{pythonin}{python}
l = [1, 2, 3, 4]
  \end{pythonin}
  \\
  \begin{pythonin}{python}
l[3] = 7
  \end{pythonin}
  \\
  \begin{pythonin}{python}
l.append(1)
  \end{pythonin}
  \\
  \begin{pythonin}{python}
l[1:3] = [3, 2]
  \end{pythonin}
  \\
  \begin{pythonin}{python}
l.sort()
  \end{pythonin}
  \\
  \begin{pythonin}{python}
l.reverse()
  \end{pythonin}
  \\
  \begin{pythonin}{python}
l
  \end{pythonin}
  \\
  \begin{pythonout}
[7, 3, 2, 1, 1]
  \end{pythonout}
 \end{ipython}
\end{pframe}


\subsection{Additional useful built-ins}
\begin{pframe}
 \begin{ipython}
  \begin{pythonin}{python}
list('abcdefghijk')
  \end{pythonin}
  \\
  \begin{pythonout}
['a', 'b', 'c', 'd', 'e', 'f', 'g', 'h', 'i', 'j', 'k']
  \end{pythonout}
  \\
  
  \begin{pythonin}{python}
range(5, 16)  # In python 2: [5, 6, 7, 8, 9, 10, 11, 12, 13, 14, 15]
  \end{pythonin}
  \\
  \begin{pythonout}
range(5, 16)
  \end{pythonout}
  \\
  
  \begin{pythonin}{python}
list(range(5, 16))
  \end{pythonin}
  \\
  \begin{pythonout}
[5, 6, 7, 8, 9, 10, 11, 12, 13, 14, 15]
  \end{pythonout}
  \\  
  
  \begin{pythonin}{python}
zip(['red', 'white', 'blue'], range(3))
  \end{pythonin}
  \\
  \begin{pythonout}
<zip at 0x7f3565860108>
  \end{pythonout}
  \\
    
  \begin{pythonin}{python}
list(zip(['red', 'white', 'blue'], range(3)))
  \end{pythonin}
  \\
  \begin{pythonout}
[('red', 0), ('white', 1), ('blue', 2)]
  \end{pythonout}
 \end{ipython}
\end{pframe}


\section{Dictionaries}

\subsection{Dictionaries map hashable values to arbitrary objects}
\begin{pframe}
 \begin{ipython}
  \begin{pythonin}{python}
d = {'a': 27, 'b': 18, 'c': 12}
  \end{pythonin}
  \\
  \begin{pythonin}{python}
type(d)
  \end{pythonin}
  \\
  \begin{pythonout}
dict
  \end{pythonout}
  \\

  \begin{pythonin}{python}
d['e'] = 17
  \end{pythonin}
  \\
  \begin{pythonin}{python}
'e' in d
  \end{pythonin}
  \\
  \begin{pythonout}
True
  \end{pythonout}
  \\

  \begin{pythonin}{python}
d.update({'a': 18, 'f': 2})
  \end{pythonin}
  \\
  \begin{pythonin}{python}
d
  \end{pythonin}
  \\
  \begin{pythonout}
{'a': 18, 'b': 18, 'c': 12, 'e': 17, 'f': 2}
  \end{pythonout}
 \end{ipython}
% \begin{itemize}
%  \item All built-in immutable objects are hashable.
%  \item No built-in mutable objects are hashable.
% \end{itemize}
\end{pframe}


\subsection{Accessing dictionary content}
\begin{pframe}
 \begin{ipython}
  \begin{pythonin}{python}
d['b']
  \end{pythonin}
  \\
  \begin{pythonout}
18
  \end{pythonout}
  \\

  \begin{pythonin}{python}
d.keys()
  \end{pythonin}
  \\
  \begin{pythonout}
dict\_keys(['e', 'c', 'a', 'b', 'f'])
  \end{pythonout}
  \\

  \begin{pythonin}{python}
list(d.keys())
  \end{pythonin}
  \\
  \begin{pythonout}
['a', 'c', 'b', 'e', 'f']
  \end{pythonout}
  \\

  \begin{pythonin}{python}
list(d.values())
  \end{pythonin}
  \\
  \begin{pythonout}
[18, 12, 18, 17, 2]
  \end{pythonout}
  \\

  \begin{pythonin}{python}
list(d.items())
  \end{pythonin}
  \\
  \begin{pythonout}
[('a', 18), ('c', 12), ('b', 18), ('e', 17), ('f', 2)]
  \end{pythonout}
 \end{ipython}
\end{pframe}



\section{Sets}

\subsection{Mutable unordered collections of hashable values without duplication}
\begin{pframe}
 \begin{ipython}
  \begin{pythonin}{python}
x = {12, 28, 21, 17}
  \end{pythonin}
  \\
  \begin{pythonin}{python}
type(x)
  \end{pythonin}
  \\
  \begin{pythonout}
set
  \end{pythonout}
  \\

  \begin{pythonin}{python}
x.add(12)
  \end{pythonin}
  \\
  \begin{pythonin}{python}
a
  \end{pythonin}
  \\
  \begin{pythonout}
{12, 17, 21, 28}
  \end{pythonout}
  \\

  \begin{pythonin}{python}
x.discard(21)
  \end{pythonin}
  \\
  \begin{pythonin}{python}
x
  \end{pythonin}
  \\
  \begin{pythonout}
{12, 17, 28}
  \end{pythonout}
 \end{ipython}
\end{pframe}

\begin{pframe}
 \begin{ipython}
  \begin{pythonin}{python}
x[0]
  \end{pythonin}
  \begin{pythonerr}{python}
---------------------------------------------------------------------------
TypeError                                 Traceback (most recent call last)
<ipython-input-62-2f755f117ac9> in <module>()
----> 1 x[0]

TypeError: 'set' object does not support indexing
  \end{pythonerr}
 \end{ipython}
\end{pframe}



\subsection{Operations}
\begin{pframe}
 We can test for membership and apply many common set operations\\
 such as union and intersect.
 \medskip

 \begin{ipython}
  \begin{pythonin}{python}
17 in {12, 28, 21, 17}
  \end{pythonin}
  \\
  \begin{pythonout}
True
  \end{pythonout}
  \\

  \begin{pythonin}{python}
{12, 28, 21, 17} | {12, 18, 11}
  \end{pythonin}
  \\
  \begin{pythonout}
{11, 12, 17, 18, 21, 28}
  \end{pythonout}
  \\

  \begin{pythonin}{python}
{12, 28, 21, 17} & {12, 18, 11}
  \end{pythonin}
  \\
  \begin{pythonout}
{12}
  \end{pythonout}
 \end{ipython}
\end{pframe}


\subsection{Operations}
\begin{pframe}
 Difference
 \medskip

 \begin{ipython}
  \begin{pythonin}{python}
s1 = {12, 28, 21, 17}
  \end{pythonin}
  \\
  
  \begin{pythonin}{python}
s2 = {28, 32, 71, 12}
  \end{pythonin}
  \\
  
  \begin{pythonin}{python}
s1.difference(s2)
  \end{pythonin}
  \\
  \begin{pythonout}
{17, 21}
  \end{pythonout}
 \end{ipython}
\end{pframe}



\section{Booleans}

\begin{pframe}
  The two boolean values are written \mintinline{python}{False} and
  \mintinline{python}{True}.
 \begin{ipython}
  \begin{pythonin}{python}
True or False
  \end{pythonin}
  \\
  \begin{pythonout}
True
  \end{pythonout}
  \\

  \begin{pythonin}{python}
True and False
  \end{pythonin}
  \\
  \begin{pythonout}
False
  \end{pythonout}
  \\

  \begin{pythonin}{python}
not False
  \end{pythonin}
  \\
  \begin{pythonout}
True
  \end{pythonout}
 \end{ipython}
\end{pframe}


\subsection{Comparisons}
\begin{pframe}
  Comparisons can be done on all objects and return a boolean value.
 \begin{ipython}
  \begin{pythonin}{python}
22 * 3 > 66
  \end{pythonin}
  \\
  \begin{pythonout}
False
  \end{pythonout}
 \end{ipython}

 We have two equivalence relations: value equality (\mintinline{python}{==}) and
 object identity (\mintinline{python}{is}).
 \begin{ipython}
  \begin{pythonin}{python}
a, b = [1, 2, 3], [1, 2, 3]
  \end{pythonin}
  \\
  \begin{pythonin}{python}
a == b
  \end{pythonin}
  \\
  \begin{pythonout}
True
  \end{pythonout}
  \\

  \begin{pythonin}{python}
a is b
  \end{pythonin}
  \\
  \begin{pythonout}
False
  \end{pythonout}
 \end{ipython}
\end{pframe}


\section{Hands on!}
\begin{pframe}
 \vspace{-0.5cm}
 \begin{enumerate}
  \item Make a list \texttt{l1} with 10 integer elements.
  \begin{enumerate}[a]
  \item What is the sum of all the items in the \texttt{l1} list.
  \item Make a new list \texttt{l2} from \texttt{l1} that does not include the 0th, 4th, and 5th elements.
  \item Sum only the elements from \texttt{l1} which are between the 2nd and the 6th elements.
  \end{enumerate}
  \item Food:
  \begin{enumerate}[a.]
   \item Create a dictionary for food products called \texttt{prices} and put some
   values in it, e.g., "apples": 2, "oranges": 1.5, "pears": 3, ...
   \item Create a corresponding dictionary called "stocks" and put the stock
   values in it, e.g., "apples": 0, "oranges": 1, "pears": 10, ...
   \item Add another entry in the \texttt{prices} dictionary with key 'bananas' and value 13.
   \item Add another entry in the \texttt{stocks} dictionary with key 'bananas' and value 11.
   \item What is the total money value for the "bananas" (stock $\times$ price)?
   \item How many products are in the \texttt{stocks} dictionary?\\
   \item Are the number of products in the \texttt{stocks} and \texttt{prices} dictionaries equal?
   \item Are there the same products in the \texttt{stocks} and \texttt{prices} dictionaries?
   \item What is the most expensive value in the \texttt{prices} dictionary?
  \end{enumerate}
%  \item max(['a', 100, 20])
 \end{enumerate}
\end{pframe}


% Make the acknowledgements slide.
\makeAcknowledgementsSlide{
  \begin{tabular}{ll}
    Martijn Vermaat\\
    Jeroen Laros\\
    Jonathan Vis
  \end{tabular}
}

\end{document}
