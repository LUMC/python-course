\documentclass[aspectratio=1610,slidestop]{beamer}

% TODO: mutable vs immutable
% TODO: add some questions in between?


\author{Mark Santcroos}
\title{Python Programming}
\providecommand{\mySubTitle}{Data Types}
\providecommand{\myConference}{Programming Course}
\providecommand{\myDate}{26-11-2019}
\providecommand{\myGroup}{}
\providecommand{\myDepartment}{}
\providecommand{\myCenter}{}

\usetheme{lumc}

\usepackage{minted}
\usepackage{tikz}
\usepackage[many]{tcolorbox}

\definecolor{monokaibg}{HTML}{272822}
\definecolor{emailc}{HTML}{1e90FF}
\definecolor{scriptback}{HTML}{CDECF0}
\definecolor{ipyout}{HTML}{F0FFF0}

\newenvironment{ipython}
 {\begin{tcolorbox}[title=IPython,
                   title filled=false,
                   fonttitle=\scriptsize,
                   fontupper=\footnotesize,
                   enhanced,
                   colback=monokaibg,
                   drop small lifted shadow,
                   boxrule=0.1mm,
                   left=0.1cm,
                   arc=0mm,
                   colframe=black]}
 {\end{tcolorbox}}



\newenvironment{terminal}
 {\begin{tcolorbox}[title=terminal,
                   title filled=false,
                   fonttitle=\scriptsize,
                   fontupper=\footnotesize,
                   enhanced,
                   colback=monokaibg,
                   drop small lifted shadow,
                   boxrule=0.1mm,
                   left=0.1cm,
                   arc=0mm,
                   colframe=black]}
 {\end{tcolorbox}}


\newcommand{\hrefcc}[2]{\textcolor{#1}{\href{#2}{#2}}}
\newcommand{\hrefc}[3]{\textcolor{#1}{\href{#2}{#3}}}

\newcounter{cntr}
\renewcommand{\thecntr}{\texttt{[\arabic{cntr}]}}

\newenvironment{pythonin}[1]
{\VerbatimEnvironment
  \begin{minipage}[t]{0.11\linewidth}
   \textcolor{green}{\texttt{{\refstepcounter{cntr}In \thecntr:}}}
  \end{minipage}%
  \begin{minipage}[t]{0.89\linewidth}%
  \begin{minted}[
    breaklines=true,style=monokai]{#1}}
 {\end{minted}
 \end{minipage}}

\newenvironment{pythonout}
{%
  \addtocounter{cntr}{-1}
  \begin{minipage}[t]{0.11\linewidth}
   \textcolor{red}{\texttt{{\refstepcounter{cntr}Out\thecntr:}}}
  \end{minipage}%
  \color{ipyout}%
  \ttfamily%
  \begin{minipage}[t]{0.89\linewidth}%
}
{\end{minipage}}

\newenvironment{pythonerr}[1]
{\VerbatimEnvironment
  \begin{minted}[
    breaklines=true,style=monokai]{#1}}
 {\end{minted}}


\newenvironment{pythonfile}[1]
 {\begin{tcolorbox}[title=#1,
                    title filled=false,
                    coltitle=LUMCDonkerblauw,
                    fonttitle=\scriptsize,
                    fontupper=\footnotesize,
                    enhanced,
                    drop small lifted shadow,
                    boxrule=0.1mm,
                    leftrule=5mm,
                    rulecolor=white,
                    left=0.1cm,
                    colback=white!92!black,
                    colframe=scriptback]}
 {\end{tcolorbox}}


\newenvironment{pythoncode}
 {\begin{tcolorbox}[title filled=false,
                    coltitle=LUMCDonkerblauw,
                    fonttitle=\scriptsize,
                    fontupper=\footnotesize,
                    enhanced,
                    drop small lifted shadow,
                    boxrule=0.1mm,
                    leftrule=5mm,
                    rulecolor=white,
                    left=0.1cm,
                    colback=white!92!black,
                    colframe=scriptback]}
 {\end{tcolorbox}}


\newenvironment{pythonoutnonumber}
{%
  \color{ipyout}%
  \ttfamily%
}
{}


\usepackage[pyconbanner=none,pygopt={style=friendly, texcomments=true, mathescape=false}]{pythontex}
\usepackage{mdframed}
\surroundwithmdframed[backgroundcolor=black!15]{pyconsole}


\begin{document}

% This disables the \pause command, handy in the editing phase.
%\renewcommand{\pause}{}

% Make the title slide.
\makeTitleSlide{\includegraphics[height=3.5cm]{../../images/Python.pdf}}

% First page of the presentation.
\section{Introduction}
%\makeTableOfContents



\section{Sequence types}

\subsection{Lists}
\begin{pframe}
Mutable sequences of values.
\begin{pyconsole}
l = [2, 5, 2, 3, 7]
type(l)
\end{pyconsole}
\medskip
\medskip
Lists can be heterogeneous, but we typically don't use that.
\begin{pyconsole}
a = 'xyz'
[3, 'abc', 1.3e20, [a, a, 2]]
\end{pyconsole}
\end{pframe}


\subsection{Tuples}
\begin{pframe}
Immutable sequences of values.
\begin{pyconsole}
t = 'white', 77, 1.5
type(t)

color, width, scale = t
width
\end{pyconsole}
\end{pframe}


\subsection{Strings}
\begin{pframe}
Immutable sequences of characters.
\begin{pyconsole}
'a string can be written in single quotes'
\end{pyconsole}
\medskip
\medskip
Strings can also be written with double quotes, or over multiple lines with
triple-quotes.
\begin{pyconsole}
"this makes it easier to use the ' character"
\end{pyconsole}
\medskip
\begin{pyconsole}
"""A multiline string.
You see? I continued after a blank line."""
\end{pyconsole}
\end{pframe}

\begin{pframe}
But not mix them!
\begin{pyconsole}
'a string can not be written with mixed quotes"
\end{pyconsole}
\end{pframe}

\begin{pframe}
\vspace{-0.3cm}
A common operation is formatting strings using argument substitutions.
\begin{pyconsole}
'{} times {} equals {:.2f}'\
    .format('pi', 2, 6.283185307179586)
\end{pyconsole}
\medskip
\medskip
Accessing arguments by position or name is more readable.
\begin{pyconsole}
'{1} times {0} equals {2:.2f}'\
    .format('pi', 2, 6.283185307179586)

'{number} times {amount} equals {result:.2f}'\
    .format(number='pi', amount=2, result=6.283185307179586)
\end{pyconsole}
\end{pframe}


\subsection{Common sequence operations}
\begin{pframe}
All sequence types support: concatenation, membership/substring tests,
indexing, and slicing.
\begin{pyconsole}
[1, 2, 3] + [4, 5, 6]

'hay' in 'haystack'

'needle' in 'haystack'

'abcdefghijkl'[3]
\end{pyconsole}
\end{pframe}


\subsection{Slicing}
\begin{pframe}
 Slice \mintinline{python}{s} from \mintinline{python}{i} to
 \mintinline{python}{j} with \mintinline{python}{s[i:j]}.
\begin{pyconsole}
'abcdefghijkl'[4:8]
'abcdefghijkl'[:3]
\end{pyconsole}
\medskip
\medskip
 We can also define the step \mintinline{python}{k} with
 \mintinline{python}{s[i:j:k]}.
\begin{pyconsole}
'abcdefghijkl'[7:3:-1]
\end{pyconsole}
\end{pframe}


\subsection{Several helpful builtins}
\begin{pframe}
\begin{pyconsole}
len('attacgataggcatccgt')

max([17, 86, 34, 51])

sum([17, 86, 34, 51])

('atg', 22, True, 'atg').count('atg')
\end{pyconsole}
\end{pframe}


\subsection{More with lists}
\begin{pframe}
 We can replace, add, remove, reverse and sort items in-place.
\begin{pyconsole}
l = [1, 2, 3, 4]
l[3] = 7
l.append(1)
l[1:3] = [3, 2]
l.sort()
l.reverse()
l
\end{pyconsole}
\end{pframe}


\subsection{Additional useful built-ins}
\begin{pframe}
\vspace{-0.25cm}
\begin{pyconsole}
list('abcdefghijk')

range(5, 16)

list(range(5, 16))

zip(['red', 'white', 'blue'], range(3))

list(zip(['red', 'white', 'blue'], range(3)))
\end{pyconsole}
\end{pframe}


\section{Dictionaries}
\subsection{Unordered map of hashable values to arbitrary objects}
\begin{pframe}
\begin{pyconsole}
d = {'a': 27, 'b': 18, 'c': 12}
type(d)

d['e'] = 17
'e' in d

d.update({'a': 18, 'f': 2})
d
\end{pyconsole}
\end{pframe}


\subsection{Accessing dictionary content}
\begin{pframe}
\vspace{-0.25cm}
\begin{pyconsole}
d['b']

d.keys()

list(d.keys())

list(d.values())

list(d.items())
\end{pyconsole}
\end{pframe}


\section{Sets}

\subsection{Mutable unordered collections of hashable values without duplication}
\begin{pframe}
\begin{pyconsole}
x = {12, 28, 21, 17}
type(x)

x.add(12)
x

x.discard(21)
x
\end{pyconsole}
\end{pframe}

\subsection{Sets are not indexed}
\begin{pframe}
Sets are unordered collections, and therefore without index.
\begin{pyconsole}
x[0]
\end{pyconsole}
\medskip
\medskip

In contrast to lists ...
\begin{pyconsole}
list(x)[0]
\end{pyconsole}
\end{pframe}



\subsection{Operations}
\begin{pframe}
We can test for membership and apply many common set operations\\
such as union and intersect.
\medskip

\begin{pyconsole}
17 in {12, 28, 21, 17}

{12, 28, 21, 17} | {12, 18, 11}

{12, 28, 21, 17} & {12, 18, 11}
\end{pyconsole}
\end{pframe}


\subsection{Operations}
\begin{pframe}
 Difference
 \medskip

\begin{pyconsole}
s1 = {12, 28, 21, 17}

s2 = {28, 32, 71, 12}

s1.difference(s2)
\end{pyconsole}
\end{pframe}


\section{Booleans}
\begin{pframe}
The two boolean values are written \mintinline{python}{False} and
\mintinline{python}{True}.
\begin{pyconsole}
True or False

True and False

not False
\end{pyconsole}
\end{pframe}


\subsection{Comparisons}
\begin{pframe}
  Comparisons can be done on all objects and return a boolean value.
\begin{pyconsole}
1 < 2

1 == 2

"Left" == "Right"

"Right" == "Right"
\end{pyconsole}
\end{pframe}


\section{Equivalence}
\subsection{Value vs object}
\begin{pframe}
We have two equivalence relations: value equality (\mintinline{python}{==}) and
object identity (\mintinline{python}{is}).
\begin{pyconsole}
a, b = [1, 2, 3], [1, 2, 3]
a == b

a is b

a = 0
b = 0
a is b

\end{pyconsole}
\end{pframe}


\section{Casting}
\subsection{Changing the type of a value}
\begin{pframe}
Sometimes you might want to combine values of different types.
\begin{pyconsole}
x = 1
name = 'John'
name + x
\end{pyconsole}
\end{pframe}


\section{Casting}
\subsection{Combining different types}
\begin{pframe}
\begin{pyconsole}
x = 1
name = 'John'
name + str(x)
\end{pyconsole}
\medskip
\medskip
And further ...
\begin{pyconsole}
x = 1
x
str(x)
int(str(x))
\end{pyconsole}
\end{pframe}


\section{Hands on!}
\begin{pframe}
 \vspace{-0.5cm}
 \begin{enumerate}
  \item Make a list \pyv{list1} with 10 integer elements.
  \begin{enumerate}[a]
  \item What is the sum of all the items in the \pyv{list1} list.
  \item Make a list \pyv{list2} from \pyv{list1} that does not include the 0th, 4th, and 5th elements.
  \item Sum only the elements from \pyv{list1} which are between the 2nd and the 6th elements.
  \end{enumerate}
  \item Food:
  \begin{enumerate}[a.]
   \item Create a dictionary for food products called \pyv{prices} and put some
       values in it, e.g., \textit{apples}: \textbf{2}, \textit{oranges}: \textbf{1.5},
          \textit{pears}: \textbf{3}, ...
      \item Create a corresponding dictionary called \textit{stock} and put the stock
          values in it, e.g., \textit{apples}: \textbf{0}, \textit{oranges}:
          \textbf{1}, \textit{pears}: \textbf{10}, ...
   \item Add another entry in the \pyv{prices} dictionary with key \textit{bananas} and value \textbf{13}.
   \item Add another entry in the \pyv{stocks} dictionary with key \textit{bananas} and value \textbf{11}.
   \item What is the total money value for the \textit{bananas} (\pyv{stock} $\times$ \pyv{price})?
   \item How many products are in the \pyv{stocks} dictionary?\\
   \item Are the number of products in the \pyv{stocks} and \pyv{prices} dictionaries equal?
   \item Are there the same products in the \pyv{stocks} and \pyv{prices} dictionaries?
   \item What is the most expensive value in the \pyv{prices} dictionary?
  \end{enumerate}
%  \item max(['a', 100, 20])
 \end{enumerate}
\end{pframe}


% Make the acknowledgements slide.
\makeAcknowledgementsSlide{
  \begin{tabular}{ll}
    Mihai Lefter\\
    Martijn Vermaat\\
    Jeroen Laros\\
    Jonathan Vis
  \end{tabular}
}

\end{document}
