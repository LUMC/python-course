\documentclass[aspectratio=1610,slidestop]{beamer}

\author{Mihai Lefter}
\title{Python Programming}
\providecommand{\mySubTitle}{String methods, error and exceptions}
\providecommand{\myConference}{Programming Course}
\providecommand{\myDate}{1-12-2022}
\providecommand{\myGroup}{}
\providecommand{\myDepartment}{}
\providecommand{\myCenter}{}

\usetheme{lumc}

\usepackage{minted}
\usepackage{tikz}
\usepackage[many]{tcolorbox}

\definecolor{monokaibg}{HTML}{272822}
\definecolor{emailc}{HTML}{1e90FF}
\definecolor{scriptback}{HTML}{CDECF0}
\definecolor{ipyout}{HTML}{F0FFF0}

\newenvironment{ipython}
 {\begin{tcolorbox}[title=IPython,
                   title filled=false,
                   fonttitle=\scriptsize,
                   fontupper=\footnotesize,
                   enhanced,
                   colback=monokaibg,
                   drop small lifted shadow,
                   boxrule=0.1mm,
                   left=0.1cm,
                   arc=0mm,
                   colframe=black]}
 {\end{tcolorbox}}



\newenvironment{terminal}
 {\begin{tcolorbox}[title=terminal,
                   title filled=false,
                   fonttitle=\scriptsize,
                   fontupper=\footnotesize,
                   enhanced,
                   colback=monokaibg,
                   drop small lifted shadow,
                   boxrule=0.1mm,
                   left=0.1cm,
                   arc=0mm,
                   colframe=black]}
 {\end{tcolorbox}}


\newcommand{\hrefcc}[2]{\textcolor{#1}{\href{#2}{#2}}}
\newcommand{\hrefc}[3]{\textcolor{#1}{\href{#2}{#3}}}

\newcounter{cntr}
\renewcommand{\thecntr}{\texttt{[\arabic{cntr}]}}

\newenvironment{pythonin}[1]
{\VerbatimEnvironment
  \begin{minipage}[t]{0.11\linewidth}
   \textcolor{green}{\texttt{{\refstepcounter{cntr}In \thecntr:}}}
  \end{minipage}%
  \begin{minipage}[t]{0.89\linewidth}%
  \begin{minted}[
    breaklines=true,style=monokai]{#1}}
 {\end{minted}
 \end{minipage}}

\newenvironment{pythonout}
{%
  \addtocounter{cntr}{-1}
  \begin{minipage}[t]{0.11\linewidth}
   \textcolor{red}{\texttt{{\refstepcounter{cntr}Out\thecntr:}}}
  \end{minipage}%
  \color{ipyout}%
  \ttfamily%
  \begin{minipage}[t]{0.89\linewidth}%
}
{\end{minipage}}

\newenvironment{pythonerr}[1]
{\VerbatimEnvironment
  \begin{minted}[
    breaklines=true,style=monokai]{#1}}
 {\end{minted}}


\newenvironment{pythonfile}[1]
 {\begin{tcolorbox}[title=#1,
                    title filled=false,
                    coltitle=LUMCDonkerblauw,
                    fonttitle=\scriptsize,
                    fontupper=\footnotesize,
                    enhanced,
                    drop small lifted shadow,
                    boxrule=0.1mm,
                    leftrule=5mm,
                    rulecolor=white,
                    left=0.1cm,
                    colback=white!92!black,
                    colframe=scriptback]}
 {\end{tcolorbox}}


\newenvironment{pythoncode}
 {\begin{tcolorbox}[title filled=false,
                    coltitle=LUMCDonkerblauw,
                    fonttitle=\scriptsize,
                    fontupper=\footnotesize,
                    enhanced,
                    drop small lifted shadow,
                    boxrule=0.1mm,
                    leftrule=5mm,
                    rulecolor=white,
                    left=0.1cm,
                    colback=white!92!black,
                    colframe=scriptback]}
 {\end{tcolorbox}}


\newenvironment{pythonoutnonumber}
{%
  \color{ipyout}%
  \ttfamily%
}
{}



\begin{document}

% This disables the \pause command, handy in the editing phase.
%\renewcommand{\pause}{}

% Make the title slide.
\makeTitleSlide{\includegraphics[height=3.5cm]{../../images/Python.pdf}}

% First page of the presentation.
\section{Introduction}
\makeTableOfContents

\subsection{Let's start with a simple GC calculator}
\begin{pframe}
 \begin{pythonfile}{seq\_toolbox.py}
  \begin{minted}[linenos]{python}
def calc_gc_percent(seq):
    at_count, gc_count = 0, 0
    for char in seq:
        if char in ('A', 'T'):
            at_count += 1
        elif char in ('G', 'C'):
            gc_count += 1

    return gc_count * 100.0 / (gc_count + at_count)

print("The sequence 'CAGG' has a %GC of {:.2f}".format(
          calc_gc_percent("CAGG")))
  \end{minted}
 \end{pythonfile}
 \pause
 Our script is nice and dandy, but we don't want to edit the source file
 everytime we calculate a sequence's GC.
\end{pframe}



\section{The standard library}

\begin{pframe}
\begin{itemize}
  \item A collection of Python modules (or functions, for now) that comes
  packaged with a default Python installation.
  \item They're not part of the language per se, more like a batteries included
  thing.
\end{itemize}

\end{pframe}

\subsection{Our first standard library module: sys}
\begin{pframe}
 \begin{itemize}
  \item We'll start by using the simple sys module to make our script more
  flexible.
  \item Standard library (and other modules, as we'll see later) can be used
  via the import statement, for example:
 \end{itemize}

 \begin{ipython}
   \begin{pythonin}{python}
import sys
   \end{pythonin}
 \end{ipython}

\begin{itemize}
 \item Like other objects so far, we can peek into the documentation of these
 modules using help, or the IPython ? shortcut. For example:
\end{itemize}
 \begin{ipython}
   \begin{pythonin}{python}
sys?
   \end{pythonin}
 \end{ipython}
\end{pframe}


\subsection{The sys.argv list}
\begin{pframe}
  \begin{itemize}
   \item The sys module allows to capture command line arguments with its argv
  object.
  \item This is a list of arguments supplied when invoking the current Python
  session.
  \item Not really useful for an interpreter session, but very handy for
  scripts.
  \end{itemize}
 \begin{ipython}
   \begin{pythonin}{python}
sys.argv
   \end{pythonin}
   \\
   \begin{pythonout}
['/usr/local/bin/ipython']
   \end{pythonout}
 \end{ipython}
\end{pframe}


\subsection{Improving our script with sys.argv}
\begin{pframe}
 \begin{pythonfile}{seq\_toolbox.py}
  \begin{minted}[linenos]{python}
import sys

def calc_gc_percent(seq):
    at_count, gc_count = 0, 0
    for char in seq:
        if char in ('A', 'T'):
            at_count += 1
        elif char in ('G', 'C'):
            gc_count += 1

    return gc_count * 100.0 / (gc_count + at_count)

input_seq = sys.argv[1]
print("The sequence '{}' has a %GC of {:.2f}".format(
          input_seq, calc_gc_percent(input_seq)))
  \end{minted}
 \end{pythonfile}
\end{pframe}



\section{String methods}

\begin{pframe}

 \begin{itemize}
  \vspace{-0.8cm}
  \item Try running the script with \mintinline{python}{'cagg'} as the input
  sequence. What happens?
  \item As we saw earlier, many objects, like those of type
  \mintinline{python}{list}, \mintinline{python}{dict}, or
  \mintinline{python}{str}, have useful methods defined on them.
  \item One way to squash this potential bug is by using Python's string method
  upper.
  \item Let's first check out some commonly used string functions.
 \end{itemize}

 \begin{ipython}
   \begin{pythonin}{python}
my_str = 'Hello again, ipython!'
   \end{pythonin}
   \\
   \begin{pythonin}{python}
my_str.upper()
   \end{pythonin}
   \\
   \begin{pythonout}
'HELLO AGAIN, IPYTHON!'
   \end{pythonout}
   \\

   \begin{pythonin}{python}
my_str.lower()
   \end{pythonin}
   \\
   \begin{pythonout}
'hello again, ipython!'
   \end{pythonout}
   \\

   \begin{pythonin}{python}
my_str.title()
   \end{pythonin}
   \\
   \begin{pythonout}
'Hello Again, Ipython!'
   \end{pythonout}
  \end{ipython}
\end{pframe}

\begin{pframe}
 \begin{ipython}
   \begin{pythonin}{python}
my_str.startswith('H')
   \end{pythonin}
   \\
   \begin{pythonout}
True
   \end{pythonout}
   \\

   \begin{pythonin}{python}
my_str.startswith('h')
   \end{pythonin}
   \\
   \begin{pythonout}
False
   \end{pythonout}
   \\

   \begin{pythonin}{python}
my_str.split(',')
   \end{pythonin}
   \\
   \begin{pythonout}
['Hello again', ' ipython!']
   \end{pythonout}
   \\

   \begin{pythonin}{python}
my_str.replace('ipython', 'lumc')
   \end{pythonin}
   \\
   \begin{pythonout}
'Hello again, lumc!'
   \end{pythonout}
   \\

   \begin{pythonin}{python}
my_str.count('n')
   \end{pythonin}
   \\
   \begin{pythonout}
2
   \end{pythonout}
 \end{ipython}
\end{pframe}




\subsection{Improving our script with upper()}
\begin{pframe}
 \begin{pythonfile}{seq\_toolbox.py}
  \begin{minted}[linenos]{python}
import sys

def calc_gc_percent(seq):
    at_count, gc_count = 0, 0
    for char in seq.upper():
        if char in ('A', 'T'):
            at_count += 1
        elif char in ('G', 'C'):
            gc_count += 1

    return gc_count * 100.0 / (gc_count + at_count)

input_seq = sys.argv[1]
print("The sequence '{}' has a %GC of {:.2f}".format(
          input_seq, calc_gc_percent(input_seq)))
  \end{minted}
 \end{pythonfile}
\end{pframe}


\section{Improving our script with comments and docstrings}
\begin{pframe}
  \vspace{-1.2cm}
 \begin{pythonfile}{seq\_toolbox.py}
  \begin{tiny}
  \begin{minted}[linenos]{python}
import sys

def calc_gc_percent(seq):
    """
    Calculates the GC percentage of the given sequence.

    Arguments:
        - seq - the input sequence (string).

    Returns:
        - GC percentage (float).

    The returned value is always <= 100.0
    """
    at_count, gc_count = 0, 0
    # Change input to all caps to allow for non-capital
    # input sequence.
    for char in seq.upper():
        if char in ('A', 'T'):
            at_count += 1
        elif char in ('G', 'C'):
            gc_count += 1

    return gc_count * 100.0 / (gc_count + at_count)

input_seq = sys.argv[1]
print("The sequence '{}' has a %GC of {:.2f}".format(
          input_seq, calc_gc_percent(input_seq)))
  \end{minted}
  \end{tiny}
 \end{pythonfile}
\end{pframe}



\section{Errors and exceptions}

\begin{pframe}
 \begin{itemize}
  \item Try running the script with \mintinline{python}{'ACTG123'} as the
  argument.
  \begin{itemize}
   \item What happens?
   \item Is this acceptable behavior?
  \end{itemize}
  \item Sometimes we want to put safeguards to handle invalid inputs. In this
  case we only accept ACTG, all other characters are invalid.
  \item Python provides a way to break out of the normal execution flow, by
  raising what's called as an \mintinline{python}{exception}.
  \item We can raise exceptions ourselves as well, by using the
  \mintinline{python}{raise} statement.
 \end{itemize}
\end{pframe}


\subsection{The ValueError built-in exception}

\begin{pframe}
 \begin{itemize}
  \item Used on occasions where inappropriate argument values are used,
  for example when trying to convert the string A to an integer:
 \end{itemize}
 \begin{ipython}
   \begin{pythonin}{python}
int('A')
   \end{pythonin}
   \begin{pythonerr}{python}
---------------------------------------------------------------------------
ValueError                                Traceback (most recent call last)
<ipython-input-14-0da6d315d7ad> in <module>()
----> 1 int('A')

ValueError: invalid literal for int() with base 10: 'A'
   \end{pythonerr}
 \end{ipython}
 \begin{itemize}
  \item ValueError is the appropriate exception to raise when your function is
  called with argument values it cannot handle.
 \end{itemize}
\end{pframe}


\subsection{Improving our script by handling invalid inputs}
\begin{pframe}
 \begin{pythonfile}{seq\_toolbox.py}
  \begin{tiny}
  \begin{minted}[linenos]{python}
def calc_gc_percent(seq):
    """
    Calculates the GC percentage of the given sequence.

    Arguments:
        - seq - the input sequence (string).

    Returns:
        - GC percentage (float).

    The returned value is always <= 100.0
    """
    at_count, gc_count = 0, 0
    # Change input to all caps to allow for non-capital
    # input sequence.
    for char in seq.upper():
        if char in ('A', 'T'):
            at_count += 1
        elif char in ('G', 'C'):
            gc_count += 1
        else:
            raise ValueError('Unexpected character found: {}. Only '
                             'ACTGs are allowed.'.format(char))

    return gc_count * 100.0 / (gc_count + at_count)
    \end{minted}
  \end{tiny}
 \end{pythonfile}
\end{pframe}


\subsection{Handling corner cases}
\begin{pframe}
 \begin{itemize}
  \item Try running the script with \mintinline{python}{''} as the argument.
  \begin{itemize}
   \item What happens?
   \item Why? Is this a valid input?
  \end{itemize}
  \item We don't always want to let exceptions stop program flow,
  sometimes we want to provide alternative flow.
  \item The \mintinline{python}{try} ... \mintinline{python}{except} block
allows you to do this.
 \end{itemize}
\end{pframe}



\subsection{Improving our script by handling corner cases}
\begin{pframe}
 \begin{pythonfile}{seq\_toolbox.py}
  \begin{tiny}
  \begin{minted}[linenos]{python}
def calc_gc_percent(seq):
    """
    Calculates the GC percentage of the given sequence.
    ...
    The returned value is always <= 100.0
    """
    at_count, gc_count = 0, 0
    # Change input to all caps to allow for non-capital
    # input sequence.
    for char in seq.upper():
        if char in ('A', 'T'):
            at_count += 1
        elif char in ('G', 'C'):
            gc_count += 1
        else:
            raise ValueError('Unexpected character found: {}. Only '
                             'ACTGs are allowed.'.format(char))

    # Corner case handling: empty input sequence.
    try:
        return gc_count * 100.0 / (gc_count + at_count)
    except ZeroDivisionError:
        return 0.0
  \end{minted}
  \end{tiny}
 \end{pythonfile}
\end{pframe}


\subsection{Aim for a minimal try block}
\begin{pframe}
 \vspace{-0.5cm}
 \begin{itemize}
  \item We want to be able to pinpoint the statements that may raise the
  exceptions so we can tailor our handling.
  \item Example of code that violates this principle:
 \end{itemize}
 \vspace{-0.1cm}
 \begin{pythoncode}
   \begin{minted}{python}
try:
    my_function()
    my_other_function()
except ValueError:
    my_fallback_function()
   \end{minted}
  \end{pythoncode}
 \begin{itemize}
  \item A better way would be:
 \end{itemize}
 \vspace{-0.1cm}
 \begin{pythoncode}
   \begin{minted}{python}
try:
    my_function()
except ValueError:
    my_fallback_function()
my_other_function()
   \end{minted}
  \end{pythoncode}
\end{pframe}


\subsection{Be specific when handling exceptions}
\begin{pframe}
 \vspace{-0.5cm}
 \begin{itemize}
  \item The following code is syntactically valid, but never use it:
 \end{itemize}
 \vspace{-0.1cm}
 \begin{pythoncode}
   \begin{minted}{python}
try:
    my_function()
except:
    my_fallback_function()
    \end{minted}
  \end{pythoncode}
 \begin{itemize}
  \item Always use the full exception name when to make for a much cleaner code:
 \end{itemize}
 \vspace{-0.1cm}
 \begin{pythoncode}
   \begin{minted}{python}
try:
    my_function()
except ValueError:
    my_fallback_function()
except TypeError:
    my_other_fallback_function()
except IndexError:
    my_final_function()
    \end{minted}
  \end{pythoncode}
\end{pframe}


\subsection{Look Before You Leap (LBYL) vs Easier to Ask for Apology (EAFP)}
\begin{pframe}
 \vspace{-0.5cm}
 \begin{itemize}
  \item We could have written our last exception block like so:
 \end{itemize}
 \begin{pythoncode}
   \begin{minted}{python}
if gc_count + at_count == 0:
    return 0.0
return gc_count * 100.0 / (gc_count + at_count)
    \end{minted}
  \end{pythoncode}
 \begin{itemize}
  \item Both approaches are correct and have their own plus and minuses in
  general.
 \end{itemize}
\end{pframe}

% Make the acknowledgements slide.
\makeAcknowledgementsSlide{
  \begin{tabular}{ll}
    Martijn Vermaat\\
    Jeroen Laros\\
    Jonathan Vis
  \end{tabular}
}

\end{document}
