\documentclass[aspectratio=1610,slidestop]{beamer}

\author{Mihai Lefter}
\title{Python Programming}
\providecommand{\mySubTitle}{Standard library, reading and writing files}
\providecommand{\myConference}{Programming Course}
\providecommand{\myDate}{01-12-2022}
\providecommand{\myGroup}{}
\providecommand{\myDepartment}{}
\providecommand{\myCenter}{}

\usetheme{lumc}

\usepackage{minted}
\usepackage{tikz}
\usepackage[many]{tcolorbox}

\definecolor{monokaibg}{HTML}{272822}
\definecolor{emailc}{HTML}{1e90FF}
\definecolor{scriptback}{HTML}{CDECF0}
\definecolor{ipyout}{HTML}{F0FFF0}
\definecolor{linkc}{HTML}{1e90FF}
\definecolor{urlc}{HTML}{1e90FF}


\newcommand{\emp}[1]{\colorbox{gray!30}{\textcolor{teal}{\lstinline{#1}}}}
\newcommand{\empt}[1]{\colorbox{gray!20}{\textcolor{red}{\lstinline{#1}}}}

% \newcommand{\pythoninline}[1]{\mintinline{python}{#1}}

\newcommand{\pythoninline}[1]{\colorbox{gray!15}{\mintinline{python}{#1}}}


\newenvironment{ipython}
 {\begin{tcolorbox}[title=IPython,
                   title filled=false,
                   fonttitle=\scriptsize,
                   fontupper=\footnotesize,
                   enhanced,
                   colback=monokaibg,
                   drop small lifted shadow,
                   boxrule=0.1mm,
                   left=0.1cm,
                   arc=0mm,
                   colframe=black]}
 {\end{tcolorbox}}



\newenvironment{terminal}
 {\begin{tcolorbox}[title=terminal,
                   title filled=false,
                   fonttitle=\scriptsize,
                   fontupper=\footnotesize,
                   enhanced,
                   colback=monokaibg,
                   drop small lifted shadow,
                   boxrule=0.1mm,
                   left=0.1cm,
                   arc=0mm,
                   colframe=black]}
 {\end{tcolorbox}}


\newcommand{\hrefcc}[2]{\textcolor{#1}{\href{#2}{#2}}}
\newcommand{\hrefc}[3]{\textcolor{#1}{\href{#2}{#3}}}

\newcounter{cntr}
\renewcommand{\thecntr}{\texttt{[\arabic{cntr}]}}

\newenvironment{pythonin}[1]
{\VerbatimEnvironment
  \begin{minipage}[t]{0.11\linewidth}
   \textcolor{green}{\texttt{{\refstepcounter{cntr}In \thecntr:}}}
  \end{minipage}%
  \begin{minipage}[t]{0.89\linewidth}%
  \begin{minted}[
    breaklines=true,style=monokai]{#1}}
 {\end{minted}
 \end{minipage}}

\newenvironment{pythonout}
{%
  \addtocounter{cntr}{-1}
  \begin{minipage}[t]{0.11\linewidth}
   \textcolor{red}{\texttt{{\refstepcounter{cntr}Out\thecntr:}}}
  \end{minipage}%
  \color{ipyout}%
  \ttfamily%
  \begin{minipage}[t]{0.89\linewidth}%
}
{\end{minipage}}

\newenvironment{pythonerr}[1]
{\VerbatimEnvironment
  \begin{minted}[
    breaklines=true,style=monokai]{#1}}
 {\end{minted}}


\newenvironment{pythonfile}[1]
 {\begin{tcolorbox}[title=#1,
                    title filled=false,
                    coltitle=LUMCDonkerblauw,
                    fonttitle=\scriptsize,
                    fontupper=\footnotesize,
                    enhanced,
                    drop small lifted shadow,
                    boxrule=0.1mm,
                    leftrule=5mm,
                    rulecolor=white,
                    left=0.1cm,
                    colback=white!92!black,
                    colframe=scriptback]}
 {\end{tcolorbox}}


\newenvironment{pythoncode}
 {\begin{tcolorbox}[title filled=false,
                    coltitle=LUMCDonkerblauw,
                    fonttitle=\scriptsize,
                    fontupper=\footnotesize,
                    enhanced,
                    drop small lifted shadow,
                    boxrule=0.1mm,
                    leftrule=5mm,
                    rulecolor=white,
                    left=0.1cm,
                    colback=white!92!black,
                    colframe=scriptback]}
 {\end{tcolorbox}}


\newenvironment{pythonoutnonumber}
{%
  \color{ipyout}%
  \ttfamily%
}
{}

\newenvironment{pythondeclaration}
 {\begin{tcolorbox}[title filled=false,
                    coltitle=LUMCDonkerblauw,
                    fonttitle=\scriptsize,
                    fontupper=\footnotesize,
                    enhanced,
                    drop small lifted shadow,
                    boxrule=0.1mm,
                    rulecolor=white,
                    left=0.1cm,
                    colback=white!92!black,
                    colframe=scriptback]}
 {\end{tcolorbox}}


\providecommand{\makeTableOfContentsSection}{
  \begin{frame}
    \frametitle{ }

    \tableofcontents[currentsection,hideallsubsections]
  \end{frame}
}




\begin{document}

% This disables the \pause command, handy in the editing phase.
%\renewcommand{\pause}{}

% Make the title slide.
\makeTitleSlide{\includegraphics[height=3.5cm]{../../images/Python.pdf}}

% First page of the presentation.
\section{Introduction}
\makeTableOfContents

\section{Working with modules}

\begin{pframe}
 \begin{itemize}
  \item A module allows you to share code in the form of libraries.
  \item You've seen one example: the sys module in the standard library.
  \item There are many other modules in the standard library, as we'll see soon.
 \end{itemize}
\end{pframe}

\subsection{What modules look like}
\begin{pframe}
 \begin{itemize}
  \item Any Python script can in principle be imported as a module.
  \item We can import whenever we can write a valid Python statement, in a
  script or in an interpreter session.
  \item If a script is called \texttt{script.py}, then we use
  \mintinline{python}{import script}.
  \item This gives us access to the objects defined in \texttt{script.py} by
  prefixing them with \texttt{script} and a dot.
  \item Keep in mind that this is not the only way to import Python modules.
  \item Refer to the Python documentation to find out more ways to do imports.
 \end{itemize}
\end{pframe}

\subsection{Using seq\_toolbox.py as a module}
\begin{pframe}
 Open an interpreter and try importing your module:
 \begin{ipython}
  \begin{pythonin}{python}
import seq_toolbox
  \end{pythonin}
 \end{ipython}
Does this work? Why?
\end{pframe}

\begin{pframe}
 \begin{ipython}
  \begin{pythonerr}{python}
---------------------------------------------------------------------------
IndexError                                Traceback (most recent call last)
<ipython-input-1-ccf54d4de53d> in <module>()
----> 1 import seq_toolbox

~/.../seq_toolbox.py in <module>()
     35
     36
---> 37 input_seq = sys.argv[1]
     38 print("The sequence '{}' has a %GC of {:.2f}".format(
     39           input_seq, calc_gc_percent(input_seq)))

IndexError: list index out of range
  \end{pythonerr}
 \end{ipython}
\end{pframe}

\subsection{Improving our script for importing}
\begin{pframe}
 \vspace{-.05cm}
 \begin{itemize}
  \item During a module import, Python executes all the statements inside the
  module.
  \item To make our script work as a module (in the intended way), we need to
  add a check whether the module is imported or not:
 \end{itemize}
 \vspace{-0.2cm}
 \begin{pythoncode}
   \begin{minted}{python}
if __name__ == '__main__':
    input_seq = sys.argv[1]
    print("The sequence '{}' has %GC of {:.2f}".format(
              input_seq, calc_gc_percent(input_seq)))
   \end{minted}
 \end{pythoncode}
 \begin{itemize}
  \item If the Python interpreter is running that module as the main program,
  it sets the special \mintinline{python}{__name__} variable to have a value
  \mintinline{python}{"__main__"}.
  \item If this file is being imported from another module,
  \mintinline{python}{__name__} will be set to the module's name.
  \item Now try importing the module again. What happens? Can you still use the
  module as a script?
 \end{itemize}
\end{pframe}


\subsection{Using modules}
\begin{pframe}
 \begin{itemize}
  \item When a module is imported, we can access the objects defined in it:
 \end{itemize}
 \begin{ipython}
   \begin{pythonin}{python}
import seq_toolbox
   \end{pythonin}
   \\
   \begin{pythonin}{python}
seq_toolbox.calc_gc_percent
   \end{pythonin}
   \\
   \begin{pythonout}
<function seq\_toolbox.calc\_gc\_percent>
   \end{pythonout}
 \end{ipython}
\end{pframe}

\begin{pframe}
 \begin{itemize}
  \item By the way, remember we added docstring to the calc\_gc\_percent
  function?
  \item After importing our module, we can read up on how to use the function
  in its docstring:
 \end{itemize}

 \begin{ipython}
   \begin{pythonin}{python}
seq_toolbox.calc_gc_percent?
   \end{pythonin}
   \\
   \begin{pythonin}{python}
seq_toolbox.calc_gc_percent('ACTG')
   \end{pythonin}
   \\
   \begin{pythonout}
50.0
   \end{pythonout}
 \end{ipython}
\end{pframe}

\subsection{Using modules}
\begin{pframe}
 \begin{itemize}
  \item We can also expose an object inside the module directly into our
  current namespace using the \mintinline{python}{from ... import ...}
  statement:
 \end{itemize}
 \begin{ipython}
   \begin{pythonin}{python}
from seq_toolbox import calc_gc_percent
   \end{pythonin}
   \\
   \begin{pythonin}{python}
calc_gc_percent('AAAG')
   \end{pythonin}
   \\
   \begin{pythonout}
25.0
   \end{pythonout}
 \end{ipython}
\end{pframe}

\subsection{(A simple guide on) How modules are discovered}
\begin{pframe}
 \begin{itemize}
  \item In our case, Python imports by checking whether the module exists in
  the current directory.
  \item This is not the only place Python looks, however.
  \item A complete list of paths where Python looks for modules is available
  via the sys module as sys.path. It is composed of (in order):
  \begin{itemize}
   \item The current directory.
   \item The PYTHONPATH environment variable.
   \item Installation-dependent defaults.
  \end{itemize}
 \end{itemize}
\end{pframe}


\section{More standard library}

\subsection{os module}
\begin{pframe}
 \begin{itemize}
  \item provides a portable way of using various operating system-specific
  functionality.
  \item It is a large module, but the one of the most frequently used bits is
  the file-related functions.
 \end{itemize}
 \begin{ipython}
  \begin{pythonin}{python}
import os
  \end{pythonin}
  \\
  \begin{pythonin}{python}
os.getcwd()    # Get current directory.
  \end{pythonin}
  \\
  \begin{pythonout}
'/home/student/projects/programming-course'
  \end{pythonout}
  \\

  \begin{pythonin}{python}
my_filename = 'input.fastq'
  \end{pythonin}
  \\
  \begin{pythonin}{python}
os.path.splitext(my_filename)    # Split the extension and filename.
  \end{pythonin}
  \\
  \begin{pythonout}
('input', '.fastq')
  \end{pythonout}
  \\

  \begin{pythonin}{python}
os.path.isdir('/home')    # Checks whether '/home' is a directory.
  \end{pythonin}
  \\
  \begin{pythonout}
True
  \end{pythonout}
 \end{ipython}
\end{pframe}


\subsection{math module}
\begin{pframe}
 \begin{itemize}
  \item Useful math-related functions can be found here.
  \item Other more comprehensive modules exist (numpy, your lesson tomorrow),
  but nevertheless math is still useful.
 \end{itemize}
 \begin{ipython}
  \begin{pythonin}{python}
import math
  \end{pythonin}
  \\
  \begin{pythonin}{python}
math.log(10)    # Natural log of 10.
  \end{pythonin}
  \\
  \begin{pythonout}
2.302585092994046
  \end{pythonout}
  \\

  \begin{pythonin}{python}
math.log(100, 10)    # Log base 10 of 100.
  \end{pythonin}
  \\
  \begin{pythonout}
2.0
  \end{pythonout}
  \\

  \begin{pythonin}{python}
math.sqrt(2)    # Square root of 2.
  \end{pythonin}
  \\
  \begin{pythonout}
1.4142135623730951
  \end{pythonout}
 \end{ipython}
\end{pframe}


\subsection{random module}
\begin{pframe}
 \begin{itemize}
  \item The random module contains useful functions for generating
  pseudo-random numbers.
 \end{itemize}
 \begin{ipython}
  \begin{pythonin}{python}
import random
  \end{pythonin}
  \\
  \begin{pythonin}{python}
math.log(10)    # Natural log of 10.
  \end{pythonin}
  \\
  \begin{pythonout}
0.9562916447281146
  \end{pythonout}
  \\

  \begin{pythonin}{python}
random.randint(2, 17)    # Random integer between 2 and 17, inclusive.
  \end{pythonin}
  \\
  \begin{pythonout}
13
  \end{pythonout}
  \\

  \begin{pythonin}{python}
# Random choice of any items in the given list.
random.choice(['apple', 'banana', 'grape', 'kiwi', 'orange'])
  \end{pythonin}
  \\
  \begin{pythonout}
'grape'
  \end{pythonout}
 \end{ipython}
\end{pframe}


\subsection{argparse module}
\begin{pframe}
 \begin{itemize}
  \item Using sys.argv is neat for small scripts, but as our script gets larger
  and more complex, we want to be able to handle complex arguments too.
  \item The argparse module has handy functionalities for creating command-line
  scripts.
 \end{itemize}
\end{pframe}

\begin{pframe}
 \vspace{-0.5cm}
 \begin{itemize}
  \item Open your script/module in a text editor and replace
  \mintinline{python}{import sys} with \mintinline{python}{import argparse}.
  \item Remove all lines / blocks referencing \mintinline{python}{sys.argv}.
  \item Change the \mintinline{python}{if __name__ == '__main__'} block to be
  the following:
 \end{itemize}
 \vspace{-0.2cm}
 \begin{pythoncode}
  \begin{minted}{python}
if __name__ == '__main__':
    # Create our argument parser object.
    parser = argparse.ArgumentParser()
    # Add the expected argument.
    parser.add_argument('input_seq', type=str,
                        help="Input sequence")
    # Do the actual parsing.
    args = parser.parse_args()
    # And show the output.
    print("The sequence '{}' has %GC of {:.2f}".format(
              args.input_seq,
              calc_gc_percent(args.input_seq)))
   \end{minted}
 \end{pythoncode}
\end{pframe}



\section{Working with text files}

\begin{pframe}
 \begin{itemize}
  \item Opening files for reading or writing is done using the
  \mintinline{python}{open} function.
  \item It is commonly used with two arguments, \texttt{name} and \texttt{mode}:
  \begin{itemize}
   \item \texttt{name} is the name of the file to open.
   \item \texttt{mode} specifies how the file should be handled.
  \end{itemize}
  \item These are some of the common file modes:
  \begin{itemize}
   \item r: open file for reading (default).
   \item w: open file for writing.
   \item a: open file for appending content.
  \end{itemize}
 \end{itemize}
\end{pframe}

\subsection{Reading files}
\begin{pframe}
 \begin{itemize}
  \item Let's go through some ways of reading from a file.
 \end{itemize}
 \begin{ipython}
  \begin{pythonin}{python}
fh = open('data/short_file.txt')
  \end{pythonin}
 \end{ipython}

 \begin{itemize}
  \item fh is a file handle object which we can use to retrieve the file
  contents.
  \item One simple way would be to read the whole file contents:
 \end{itemize}
 \begin{ipython}
  \begin{pythonin}{python}
fh.read()
  \end{pythonin}
  \\
  \begin{pythonout}
'this short file has two lines it is used in the example code'
  \end{pythonout}
 \end{ipython}
\end{pframe}

\begin{pframe}
 \begin{itemize}
  \item Executing fh.read() a second time gives an empty string.
  This is because we have "walked" through the file to its end.
 \end{itemize}
 \vspace{-0.1cm}
 \begin{ipython}
  \begin{pythonin}{python}
fh.read()
  \end{pythonin}
  \\
  \begin{pythonout}
''
  \end{pythonout}
 \end{ipython}
 \begin{itemize}
  \item We can reset the handle to the beginning of the file again using the
  \mintinline{python}{seek()} function.
  \item Here, we use 0 as the argument since we want to move the handle to
  position 0 (beginning of the file):
 \end{itemize}
 \vspace{-0.1cm}
 \begin{ipython}
  \begin{pythonin}{python}
fh.seek(0)
  \end{pythonin}
 \end{ipython}
\end{pframe}



\begin{pframe}
 \begin{itemize}
  \item In practice, reading the whole file into memory is not always a good
  idea.
  \item It is practical for small files, but not if our file is big (e.g.,
  bigger than our memory).
  \item In this case, the alternative is to use the
  \mintinline{python}{readline()} function.
 \end{itemize}
 \begin{ipython}
  \begin{pythonin}{python}
fh.readline()
  \end{pythonin}
  \\
  \begin{pythonout}
'this short file has two lines'
  \end{pythonout}
  \\

  \begin{pythonin}{python}
fh.readline()
  \end{pythonin}
  \\
  \begin{pythonout}
'it is used in the example code'
  \end{pythonout}
  \\

  \begin{pythonin}{python}
fh.readline()
  \end{pythonin}
  \\
  \begin{pythonout}
''
  \end{pythonout}
 \end{ipython}

\end{pframe}

\begin{pframe}
 \begin{itemize}
  \item More common in Python is to use the for loop with the file handle
  itself.
  \item Python will automatically iterate over each line.
 \end{itemize}
 \begin{pythoncode}
  \begin{minted}{python}
for line in fh:
    print(line)
  \end{minted}
 \end{pythoncode}
 \begin{itemize}
  \item Now that we're done with the file handle, we can call the
  \mintinline{python}{close()}
  method to free up any system resources still being used to keep the file open.
  \item After we closed the file, we can not use the file object anymore.
 \end{itemize}
\end{pframe}


\subsection{Writing files}
\begin{pframe}
 \begin{itemize}
  \item When writing files, we supply the w file mode explicitely:
 \end{itemize}
 \begin{ipython}
  \begin{pythonin}{python}
fw = open('data/my_file.txt', 'w')
  \end{pythonin}
 \end{ipython}
 \begin{itemize}
  \item fw is a file handle similar to the fh that we've seen previously.
  \item It is used only for writing and not reading, however.
 \end{itemize}
\end{pframe}

\begin{pframe}
 \begin{itemize}
  \item To write to the file, we use its \mintinline{python}{write()} method.
  \item Remember that Python does not add newline characters here
  \item (as opposed to when you use the print statement), so to move to a new
  \item line we have to add \textbackslash n ourselves.
 \end{itemize}
 \begin{ipython}
  \begin{pythonin}{python}
fw.write('This is my first line ')
  \end{pythonin}
  \\
  \begin{pythonin}{python}
fw.write('Still on my first line\n')
  \end{pythonin}
  \\
  \begin{pythonin}{python}
fw.write('Now on my second line')
  \end{pythonin}
 \end{ipython}
\end{pframe}

\begin{pframe}
 \begin{itemize}
  \item As with the r mode, we can close the handle when we're done with it.
  The file can then be reopened with the r mode and we can check its contents.
 \end{itemize}
 \begin{ipython}
  \begin{pythonin}{python}
fw.close()
  \end{pythonin}
 \end{ipython}
\end{pframe}


\subsection{Be cautious when using file handles}
\begin{pframe}
 \vspace{-0.2cm}
 \begin{itemize}
  \item When reading / writing files, we are interacting with external resources
  that may or may not behave as expected.
  \item For example:
  \begin{itemize}
   \item We don't always have permission to read / write a file.
   \item The file itself may not exist.
   \item We have a completely wrong idea of what's in the file.
  \end{itemize}
 \end{itemize}
 \vspace{-0.1cm}
 \begin{pythoncode}
  \begin{minted}{python}
try:
    f = open('data/short_file.txt')
    for line in f:
        print(int(line))
except ValueError:
    print('Seems there was a line we could not handle')
finally:
    f.close()
    print('We closed our file handle')
  \end{minted}
 \end{pythoncode}
\end{pframe}

\begin{pframe}
 \begin{itemize}
  \item This option is highly recommended:
 \end{itemize}
 \begin{pythoncode}
  \begin{minted}{python}
with open("welcome.txt") as f: # Use file to refer to the file object
    for line in f:
       #do something with data
  \end{minted}
 \end{pythoncode}

\end{pframe}

\subsection{Improving our script to allow input from a file}
\begin{pframe}
 \begin{itemize}
  \item The script should accept as its argument a path to a file containing
  sequences.
  \item It will then compute the GC percentage for each sequence in this file.
  \item There are at least two things we need to do:
  \begin{itemize}
   \item Change the argument parser so that it deals with a new execution mode.
   \item Add some statements to read from a file.
  \end{itemize}
 \end{itemize}
\end{pframe}

% Make the acknowledgements slide.
\makeAcknowledgementsSlide{
  \begin{tabular}{ll}
    Martijn Vermaat\\
    Jeroen Laros\\
    Jonathan Vis
  \end{tabular}
}

\end{document}
